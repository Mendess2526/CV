\documentclass{article}

\usepackage{titlesec}
\usepackage{titling}
\usepackage[margin=0.5in]{geometry}
\usepackage[colorlinks=true,urlcolor=blue]{hyperref}
\usepackage[absolute]{textpos} % showboxes
\usepackage{graphicx}

\titleformat{\section}
{\huge}
{}
{.25em}
{\bfseries}[\titlerule]

\titleformat{\subsection}
{\bfseries\Large}
{}
{.5em}
{}

\titleformat{\subsubsection}%[runin]
{\bfseries}
{\hspace{-1cm}}
{0em}
{}

\titlespacing{\subsubsection}
{2em}{.25em}{.25em}

\renewcommand{\maketitle}{\begin{center}
    {\huge\bfseries\theauthor}

    \vspace{.25em}

    \Large{Informatics Engineering Student at Instituto Superior Técnico de Lisboa}

    \vspace{.25em}

    \large{\thetitle}
\end{center}
}

\setlength{\TPHorizModule}{20mm}
\setlength{\TPVertModule}{\TPHorizModule}
\textblockorigin{10mm}{10mm}
\setlength{\parindent}{0pt}

\pagenumbering{gobble}

\begin{document}
\title{Curriculum Vitae}
\author{Pedro Mendes Félix da Costa}

\maketitle

\begin{textblock}{2} (0,1.2)
    \begin{flushright}
        \subsection{\hfill About me}
        \includegraphics[width=70px]{face.png}
        \subsubsection{\hfill Nationality}
        Portuguese
        \subsubsection{\hfill Date of Birth}
        1997--04--25
        \subsubsection{\hfill Email \& Skype}
        \href{mailto:pedro.mendes.26@gmail.com}{pedro.mendes.26\\@gmail.com}
        \subsubsection{\hfill Github}
        \href{https://github.com/Mendess2526}{Mendess2526}
        \subsubsection{\hfill LinkedIn}
        \href{https://www.linkedin.com/in/mendes2526/}{mendes2526}
        \subsubsection{\hfill Phone Number}
        +351 926 546 003
        \subsubsection{\hfill Gender}
        Male
        \subsubsection{\hfill Marital Status}
        Single
        \subsubsection{\hfill Postal Address}
        Rua de S. José n122 6drt, 4710-436 Braga, Portugal

        \subsection{\hfill Languages}
        \subsubsection{\hfill Spoken}
        Portuguese (native), English (C 2).
        \subsubsection{\hfill Programming}
        C, Rust, C++, Bash, Java, Haskell, Elixir, Python.
        \subsubsection{\hfill Query}
        SQL (mysql, postgres), NO-SQL (mongodb, neo4j).
        \subsubsection{\hfill Markup}
        \LaTeX, Markdown, Html.
        \subsection{\hfill Tools}
        Advanced Git knowledge.
        GNU core utils.
    \end{flushright}
\end{textblock}
\begin{textblock}{7.3} (2.3,1.2)

    \section{Education}
    \begin{tabular}{lp{8.4cm}r}
        2015-2019 & \textbf{Bachelors in Science of Computer Engineering} & University of Minho\\
        & Software Engineering. Grade Average: 15 &\\
        & Informatics Labs Average: 19 & \\
        & Algorithms and Complexity: 16 & \\
        & Program Calculus: 20 & \\
        & Imperative and OO programming: 18 and 19 (respectively) &\\
        & Compilers: 17 &\\
        2019-Present & \textbf{Masters in Information Systems and Computer Engineering} & IST\\
    \end{tabular}

    \section{Highlighted Projects}
    \begin{tabular}{lp{10.4cm}r}
        C / Java & \textbf{Structured Programming} & \href{https://github.com/Mendess2526/LI3_StructuredPrograming}{Github}\\
        & This project focused on writing structured code taking into account
        encapsulation, whilst being efficient for processing ``great'' volumes of
        data (Grade: 19/20). &\\
        C++ & \textbf{Computer Graphics} & \href{https://github.com/Mendess2526/CG}{Github}\\
        & A generic graphics engine, capable of rendering scenes written in XML (Grade 20/20). &\\
        Markdown & \textbf{ResumosMIEI} & \href{https://github.com/Mendess2526/ResumosMIEI}{Github}\\
        & A collection of notes written in Portuguese to help fellow students
        study the base concepts of computer science, lectured at University of
        Minho. &\\
        Rust & \textbf{PacMan} & \href{https://github.com/Mendess2526/rust-pacman}{Github}\\
        & An implementation of the classic PacMan game in rust, using openGL.\@ &\\
        Rust & \textbf{Scryfall} & \href{https://github.com/Mendess2526/scryfall-rs}{Github}\\
        & A wrapper around a REST API for fetching and searching for cards from the Magic: The Gathering\texttrademark card game. &\\
    \end{tabular}

    \section{Experience}
    \begin{tabular}{lp{8.4cm}r}
        2017 & \textbf{MIUP} & University of Minho\\
        & Inter-University Programming Marathon. &\\
        2018-Present & \textbf{Member of CaOS} & CeSIUM\\
        & CaOS is the open source group at CeSIUM, the software engineering
        student centre at Minho University. We promote open source software
        like GNU/Linux, freedom of speech and net neutrality. &\\
        2019 & \textbf{SEI} & CeSIUM \\
        & \textbf{S}emana da \textbf{E}ngenharia \textbf{I}nformática: An event
        populated with talks and workshops with various companies. I was part of
        the IT staff in the 2019 edition. &\\
    \end{tabular}

    \section{Interests}
    The GNU/Linux OS\@; Systems Programming\@; Giving Mentorship\@; Game Design.

\end{textblock}

\end{document}
